\subsection{Evolution Results}

% Population Evolution Table
\begin{table}[htbp]
\centering
\caption{Population Evolution Summary}
\label{tab:evolution_results}
\begin{tabular}{lccc}
\toprule
\textbf{Metric} & \textbf{Initial (Round 1)} & \textbf{Post-Evolution} & \textbf{Change} \\
\midrule
Population Size & 5 agents & 6+ agents & +1+ (+20\%+) \\
Agent Composition & Agent\_01-05 & Original + Evolved & Multiple generations \\
Tools in Ecosystem & $\sim$8 tools & 47+ shared tools & +39+ tools \\
Active Generations & 0 & $\geq$2 completed & Evolution active \\
Specialized Tools & Basic functions & \texttt{LiteratureReviewAutomator} & Advanced capabilities \\
 & & \texttt{AutoImageOptimizer} & \\
 & & \texttt{RateLimitMonitor} & \\
\bottomrule
\end{tabular}
\end{table}

\paragraph{Evolutionary Pressure Successfully Applied}
The system successfully triggered multiple generations of evolution, with agents numbered up to Agent\_20 observed in the logs, demonstrating that the complexity-based selection mechanism effectively identified and propagated successful traits. The evolved agents (Agent\_06 through Agent\_20) represent both mutation and crossover variants derived from top-performing original agents, indicating that the fitness evaluation based on Tool Complexity Index (TCI) scores successfully guided the evolutionary process beyond simple replacement toward genuine capability enhancement.

\paragraph{Ecosystem Expansion and Specialization}
Rather than maintaining a static population, the evolutionary process dramatically expanded the tool ecosystem from approximately 8 initial tools to over 47 specialized tools, with evolved agents contributing sophisticated capabilities like \texttt{LiteratureReviewAutomator}, \texttt{AutoImageOptimizer}, and \texttt{RateLimitMonitor}. This progression from basic data processing functions to domain-specific automation tools demonstrates that the prompt-level evolution mechanism enables emergent specialization, with each generation of agents developing increasingly complex and targeted solutions that complement rather than duplicate existing ecosystem capabilities. 