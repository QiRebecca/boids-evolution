\documentclass[10pt]{article}
\usepackage[utf8]{inputenc}
\usepackage{amsmath}
\usepackage{amsfonts}
\usepackage{amssymb}
\usepackage{graphicx}
\usepackage{booktabs}
\usepackage{multirow}
\usepackage{array}
\usepackage{float}
\usepackage{longtable}
\usepackage[margin=1in]{geometry}
\usepackage{listings}
\usepackage{xcolor}

\title{Supplementary Materials: Emergent Collaborative Intelligence from Minimal Cognitive Boids Rules}

\author{Experimental AI Research Lab}

\date{\today}

\begin{document}

\maketitle

\section{Detailed Experimental Results}

\subsection{Complete Experimental Configuration Matrix}

\begin{table}[H]
\centering
\caption{Complete Experimental Configuration Details}
\begin{tabular}{@{}llcccc@{}}
\toprule
\textbf{Exp ID} & \textbf{Topology} & \textbf{Agents} & \textbf{Steps} & \textbf{Total Tools} & \textbf{Intelligence Score} \\
\midrule
1 & Triangle & 4 & 30 & 41 & 75.0\% \\
2 & Line & 4 & 30 & 79 & 75.0\% \\
3 & Star & 4 & 30 & 79 & 100.0\% \\
4 & Triangle & 3 & 50 & 83 & 75.0\% \\
5 & Triangle & 4 & 50 & 141 & 75.0\% \\
6 & Triangle & 5 & 50 & 170 & 75.0\% \\
7 & Triangle & 6 & 50 & 233 & 75.0\% \\
8 & Triangle & 8 & 50 & 334 & 75.0\% \\
9 & Triangle & 10 & 50 & 432 & 75.0\% \\
10 & Triangle & 5 & 20 & 61 & 75.0\% \\
11 & Triangle & 5 & 50 & 170 & 75.0\% \\
12 & Triangle & 5 & 100 & 391 & 75.0\% \\
\midrule
\multicolumn{5}{r}{\textbf{Average Intelligence Score:}} & \textbf{77.1\%} \\
\bottomrule
\end{tabular}
\end{table}

\subsection{Detailed Intelligence Indicator Analysis}

\begin{table}[H]
\centering
\caption{Intelligence Indicators by Experiment Configuration}
\begin{tabular}{@{}lcccc@{}}
\toprule
\textbf{Configuration} & \textbf{Specialization} & \textbf{Collaboration} & \textbf{Ecosystem} & \textbf{Complexity} \\
\midrule
Triangle-4-30 & 0\% & 100\% & 100\% & 100\% \\
Line-4-30 & 0\% & 100\% & 100\% & 100\% \\
Star-4-30 & 100\% & 100\% & 100\% & 100\% \\
Triangle-3-50 & 0\% & 100\% & 100\% & 100\% \\
Triangle-4-50 & 0\% & 100\% & 100\% & 100\% \\
Triangle-5-50 & 0\% & 100\% & 100\% & 100\% \\
Triangle-6-50 & 0\% & 100\% & 100\% & 100\% \\
Triangle-8-50 & 0\% & 100\% & 100\% & 100\% \\
Triangle-10-50 & 0\% & 100\% & 100\% & 100\% \\
Triangle-5-20 & 0\% & 100\% & 100\% & 100\% \\
Triangle-5-50 & 0\% & 100\% & 100\% & 100\% \\
Triangle-5-100 & 100\% & 0\% & 100\% & 100\% \\
\midrule
\textbf{Success Rate} & \textbf{25.0\%} & \textbf{83.3\%} & \textbf{100.0\%} & \textbf{100.0\%} \\
\bottomrule
\end{tabular}
\end{table}

\subsection{Tool Type Distribution Analysis}

\begin{table}[H]
\centering
\caption{Tool Type Distribution Across All Experiments}
\begin{tabular}{@{}lcccc@{}}
\toprule
\textbf{Configuration} & \textbf{Data} & \textbf{Logic} & \textbf{Utility} & \textbf{Connector} \\
\midrule
Triangle-4-30 & 8 (19.5\%) & 13 (31.7\%) & 9 (22.0\%) & 11 (26.8\%) \\
Line-4-30 & 19 (24.1\%) & 18 (22.8\%) & 21 (26.6\%) & 21 (26.6\%) \\
Star-4-30 & 19 (24.1\%) & 18 (22.8\%) & 21 (26.6\%) & 21 (26.6\%) \\
Triangle-3-50 & 19 (22.9\%) & 24 (28.9\%) & 19 (22.9\%) & 21 (25.3\%) \\
Triangle-4-50 & 33 (23.4\%) & 42 (29.8\%) & 32 (22.7\%) & 34 (24.1\%) \\
Triangle-5-50 & 37 (21.8\%) & 56 (32.9\%) & 36 (21.2\%) & 41 (24.1\%) \\
Triangle-6-50 & 58 (24.9\%) & 67 (28.8\%) & 54 (23.2\%) & 54 (23.2\%) \\
Triangle-8-50 & 81 (24.3\%) & 100 (29.9\%) & 76 (22.8\%) & 77 (23.1\%) \\
Triangle-10-50 & 104 (24.1\%) & 130 (30.1\%) & 98 (22.7\%) & 100 (23.1\%) \\
Triangle-5-20 & 14 (23.0\%) & 19 (31.1\%) & 12 (19.7\%) & 16 (26.2\%) \\
Triangle-5-50 & 37 (21.8\%) & 56 (32.9\%) & 36 (21.2\%) & 41 (24.1\%) \\
Triangle-5-100 & 98 (25.1\%) & 116 (29.7\%) & 88 (22.5\%) & 89 (22.8\%) \\
\midrule
\textbf{Average} & \textbf{23.4\%} & \textbf{29.2\%} & \textbf{22.8\%} & \textbf{24.6\%} \\
\bottomrule
\end{tabular}
\end{table}

\section{Algorithmic Details}

\subsection{Boids Rule Implementation}

\subsubsection{Separation Rule}
The separation rule calculates action preferences $P_{sep}(a)$ to avoid building oversaturated tool types:

\begin{equation}
P_{sep}(build\_type) = base\_pref \times \prod_{n \in N_i} \text{avoidance\_factor}(type, n)
\end{equation}

where $\text{avoidance\_factor}(type, n)$ reduces preference if neighbor $n$ recently built tools of that type.

\subsubsection{Alignment Rule}
The alignment rule boosts preferences for actions taken by successful neighbors:

\begin{equation}
P_{align}(a) = base\_pref + \sum_{n \in successful\_neighbors} \text{boost\_factor} \times \mathbb{1}(a \in recent\_actions(n))
\end{equation}

where successful neighbors are defined as those with more tools than the current agent.

\subsubsection{Cohesion Rule}
The cohesion rule increases preference for using neighbors' tools:

\begin{equation}
P_{coh}(use\_tool) = base\_pref + \text{availability\_bonus} \times |available\_neighbor\_tools|
\end{equation}

\subsection{Action Selection Algorithm}

\begin{lstlisting}[language=Python, caption=Action Selection Pseudocode]
def select_action(agent, observations):
    # Calculate preferences from each rule
    sep_prefs = apply_separation_rule(observations)
    align_prefs = apply_alignment_rule(observations)
    coh_prefs = apply_cohesion_rule(observations)
    
    # Combine with genetic weights
    final_prefs = {}
    for action in all_actions:
        final_prefs[action] = (
            sep_prefs[action] * agent.separation_weight +
            align_prefs[action] * agent.alignment_weight +
            coh_prefs[action] * agent.cohesion_weight
        )
    
    # Weighted random selection
    return weighted_random_choice(final_prefs)
\end{lstlisting}

\section{Statistical Analysis}

\subsection{Collaboration Rate Distribution}

Across all experiments, collaboration rates (percentage of actions that involve using others' tools) show the following distribution:

\begin{itemize}
\item \textbf{Mean}: 24.3\%
\item \textbf{Median}: 26.0\%
\item \textbf{Standard Deviation}: 8.7\%
\item \textbf{Range}: 12.1\% - 35.4\%
\end{itemize}

\subsection{Tool Creation Productivity}

Tools created per agent per timestep across configurations:

\begin{itemize}
\item \textbf{Mean}: 0.67 tools/agent/step
\item \textbf{Median}: 0.71 tools/agent/step  
\item \textbf{Standard Deviation}: 0.23
\item \textbf{Range}: 0.31 - 1.03 tools/agent/step
\end{itemize}

\subsection{Specialization Metrics}

Agent specialization measured by Herfindahl-Hirschman Index (HHI) of tool type distribution per agent:

\begin{equation}
HHI_i = \sum_{t \in tool\_types} \left(\frac{count_t(i)}{total\_tools(i)}\right)^2
\end{equation}

where $HHI = 1.0$ indicates perfect specialization (only one tool type) and $HHI = 0.25$ indicates perfect balance across four types.

\begin{itemize}
\item \textbf{Mean HHI}: 0.42
\item \textbf{Agents with HHI > 0.5}: 34.7\% (indicating moderate to strong specialization)
\item \textbf{Agents with HHI > 0.7}: 12.3\% (indicating strong specialization)
\end{itemize}

\section{Network Topology Analysis}

\subsection{Connectivity Patterns}

\begin{table}[H]
\centering
\caption{Network Topology Characteristics}
\begin{tabular}{@{}lccc@{}}
\toprule
\textbf{Topology} & \textbf{Avg Degree} & \textbf{Clustering Coeff} & \textbf{Path Length} \\
\midrule
Triangle & $n-1$ & 1.0 & 1.0 \\
Line & 2.0 & 0.0 & $\frac{n+1}{3}$ \\
Star & $\frac{2(n-1)}{n}$ & 0.0 & 2.0 \\
\bottomrule
\end{tabular}
\end{table}

\subsection{Information Propagation Speed}

Analysis of how quickly successful strategies spread through different network topologies:

\begin{itemize}
\item \textbf{Triangle}: Immediate propagation (1 step) due to full connectivity
\item \textbf{Star}: 2-step propagation through hub
\item \textbf{Line}: Linear propagation ($O(n)$ steps for end-to-end)
\end{itemize}

\section{Reproducibility Information}

\subsection{Implementation Details}

\begin{itemize}
\item \textbf{Programming Language}: Python 3.8+
\item \textbf{Random Seed}: Fixed across experiments for reproducibility
\item \textbf{Simulation Framework}: Custom discrete-time multi-agent system
\item \textbf{Data Collection}: Full step-by-step logging in JSON format
\end{itemize}

\subsection{Parameter Settings}

\begin{table}[H]
\centering
\caption{Fixed Parameter Values Across All Experiments}
\begin{tabular}{@{}lc@{}}
\toprule
\textbf{Parameter} & \textbf{Value} \\
\midrule
Separation Weight & 0.4 \\
Alignment Weight & 0.3 \\
Cohesion Weight & 0.3 \\
Tool Types & 4 (data, logic, utility, connector) \\
Action History Length & 5 steps \\
Base Action Preferences & Uniform (0.25 each) \\
\bottomrule
\end{tabular}
\end{table}

\subsection{Code Availability}

The complete experimental codebase, including:
\begin{itemize}
\item Agent implementation (\texttt{src/simple\_boids\_agent.py})
\item Network simulation (\texttt{src/simple\_boids\_network.py})
\item Experiment runner (\texttt{run\_experiments.sh})
\item Analysis pipeline (\texttt{analyze\_experiments\_simple.py})
\item Raw experimental data (JSON format)
\end{itemize}

is available at: \texttt{https://github.com/ai-lab/boids-evolution}

\section{Extended Discussion}

\subsection{Limitations and Threats to Validity}

\subsubsection{Internal Validity}
\begin{itemize}
\item \textbf{Random Variation}: Limited number of runs per configuration
\item \textbf{Parameter Sensitivity}: Fixed boids weights may not be optimal
\item \textbf{Measurement Bias}: Intelligence indicators are researcher-defined proxies
\end{itemize}

\subsubsection{External Validity}
\begin{itemize}
\item \textbf{Tool Simplicity}: Real-world tools have complex dependencies
\item \textbf{Scale Limitations}: Small populations may not reflect large-scale dynamics
\item \textbf{Environment Simplicity}: No resource constraints or external pressures
\end{itemize}

\subsubsection{Construct Validity}
\begin{itemize}
\item \textbf{Intelligence Definition}: Operational definition may not capture all aspects
\item \textbf{Collaboration Measurement}: Tool usage frequency as proxy for collaboration
\item \textbf{Emergence Detection}: Threshold-based classification of emergence
\end{itemize}

\subsection{Future Experimental Directions}

\subsubsection{Parameter Space Exploration}
Systematic variation of boids rule weights $\{w_{sep}, w_{align}, w_{coh}\}$ to find optimal configurations for different objectives.

\subsubsection{Dynamic Network Topologies}
Investigation of time-varying network structures and their impact on emergence patterns.

\subsubsection{Heterogeneous Agent Populations}
Introduction of agents with different capabilities or rule variants to study diversity effects.

\subsubsection{Environmental Perturbations}
Addition of external shocks, resource constraints, or competitive pressures to test robustness.

\end{document} 